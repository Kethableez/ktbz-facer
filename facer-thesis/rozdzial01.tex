\chapter{Wstęp}
\section{Wprowadzenie}
Wraz z dynamicznym rozwojem technologii cyfrowych i rosnącą zależnością społeczeństwa od aplikacji internetowych, konieczność zapewnienia bezpieczeństwa danych i poufności informacji stała się nieodłączną częścią naszego życia online. Tradycyjne metody uwierzytelniania, takie jak hasła czy kody PIN, są narażone na różnego rodzaju zagrożenia, takie jak kradzież danych, phishing czy ataki siłowe. W odpowiedzi na te wyzwania, technologia sztucznej inteligencji (SI) w połączeniu z rozpoznawaniem twarzy zyskuje coraz większe znaczenie jako zaawansowana metoda uwierzytelniania w aplikacjach internetowych.

Rozpoznawanie twarzy oparte na sztucznej inteligencji to proces identyfikacji i weryfikacji tożsamości na podstawie analizy unikalnych cech twarzy. Wykorzystuje ono algorytmy SI do analizy wzorców i cech twarzy, takich jak kontury, proporcje, linie czy punkty charakterystyczne. Dzięki temu rozwiązaniu, możliwe jest tworzenie zaawansowanych systemów uwierzytelniania, które są trudniejsze do oszukania, ponieważ wymagają obecności fizycznej osoby i są mniej podatne na kradzież danych uwierzytelniających.
\section{Cel oraz zakres pracy}
Praca ta ma na celu przedstawienie zastosowania sztucznej inteligencji w kontekście rozpoznawania twarzy jako metody uwierzytelniania w aplikacjach internetowych. Analizując obecny stan technologii, przewagi i wady tego podejścia oraz potencjalne zagrożenia związane z jej wykorzystaniem, zaprezentujemy kompleksowy obraz tego, jakie korzyści może przynieść ta innowacyjna technologia oraz jakie wyzwania należy rozwiązać.

Zakresem omawianej pracy jest opracowanie aplikacji internetowej, która umożliwi przetestowanie różnych metod uwierzytelniania użytkowników.
\section{Motywacja oraz znaczenie badania}
Motywacją do omawianego tematu jest raptownie rosnące zainteresowanie tematem sztucznej inteligencji oraz bezpieczeństwa metod uwierzytelniania.

Znaczenie badania jest takie żeby sprawdzić czy internet jest gotowy na metody uwierzytelniania wykorzystujące sztuczną itneligencję.

\section{Zawartość pracy}
Niniejsza praca została podzielona na poszczególne rozdziały.

Rozdział 2 skupia się na teoretycznym podejściu do badanego tematu. Uwzględnia zarówno publikacje naukowe związane z metodami uwierzytelniania użytkowników, jak i teoretyczny opis wykorzystywanych narzędzi, modeli sieci neuronowych.
Rozdział 3 zawiera część inżynierską niniejszej pracy. Zawiera opis implementacji aplikacji internetowej oraz własnego modelu sieci neuronowej. Dodatkowo zawiera opis opracowanej ankiety do badań wśród internautów.
Rozdział 4 opisuje podejście do badań opracowanego tematu.
Rozdział 5 skupia się na analizie wyników badań.
Rozdział 6 ma za zadanie podsumować wyniki badań oraz wyciągnąć wnioski.

% Niniejszy dokument powstał z myślą o ujednoliceniu sposobu redagowania prac dyplomowych magisterskich i inżynierskich. Cel ten starano się osiągnąć poprzez sformułowanie reguł redakcyjnych oraz dostarczenie gotowego do wykorzystania kodu źródłowego, przygotowanego do kompilacji w systemie \LaTeX{}. W dokumencie zebrano więc zalecenia i uwagi o charakterze technicznym (dotyczące takich zagadnień, jak na przykład: formatowanie tekstu, załączanie rysunków, układ strony) oraz redakcyjnym (odnoszące się do stylu wypowiedzi, sposobów tworzenia referencji itp.). Ponadto skomentowano stworzony kod źródłowy, by na podstawie przekazanych wskazówek lepiej można było zrozumieć znaczenie użytych komend. W efekcie prezentowany dokument (a raczej zestaw dokumentów wchodzących w skład \LaTeX{}-owego projektu) pełnić może rolę szablonu, w którym wystarczy zmienić treść, aby po kompilacji uzyskać dobrze sformatowaną pracę dyplomową w~postaci dokumentu \texttt{pdf}.  



% Szablon przygotowano do kompilacji narzędziem \texttt{pdflatex} należącym do dystrybucji systemu \LaTeX. Aby skorzystać z szablonu należy wcześniej zainstalować ten system  bądź też skorzystać z usług kompilacji \LaTeX-owych źródeł dostępnych on-line (jak \texttt{OverLeaf}). Na pierwszy rzut oka kod źródłowy szablonu (w szczególności głównego dokumentu \texttt{Dyplom.tex}) może wydać się nieco skomplikowany. Zdecydowano bowiem, by zamiast tworzyć osobną klasę dokumentu lepiej będzie wykorzystać jakąś istniejącą klasę, oferującą zestaw komend ułatwiających składanie tekstu. Wybór padł na klasę \texttt{memoir}. W efekcie szablon utworzono jako sparametryzowaną instancję tej klasy. 
% Samo zaś użycie szablonu jest dość proste. Wystarczy podmienić wartości atrybutów komend użytych do zdefiniowania zawartości strony tytułowej (metadane dokumentu: tytuł, autor, promotor, kierunek, specjalność, słowa kluczowe), a później zadbać o~właściwą strukturę reszty dokumentu. 

% W szablonie zamieszczono komendy zapewniające dołączenie  do wynikowego dokumentu \texttt{pdf} metadanych z podstawowymi informacjami (tytuł, autor, temat, słowa kluczowe, data). Metadane te będą widoczne we właściwościach dokumentu, gdy zacznie się go przeglądać w jakiejś przeglądarce \texttt{pdf} (np.~\texttt{SumatraPD}F lub \texttt{Acrobat Reader}). Niestety, system \LaTeX{} nie wspiera tworzenia dostępnych plików \texttt{pdf} (zgodnych ze standardami PDF/UA, WCAG 2.0/2.1/2.2).

% Szablon pracy dyplomowej nie wiąże się bezpośrednio z tzw.\ ,,Kartą tematu pracy dyplomowej''. Karty tematów są syntetycznymi opisami, które podlegają oficjalnej procedurze zgłaszania, zatwierdzania i wybierania (finalizowanej dokonaniem wpisu studenta na kurs ,,Praca dyplomowa''). Formalnie zawartość kart tematów prac dyplomowych regulowana jest zarządzeniami odpowiedniego Dziekana. Przystępując do redakcji pracy dyplomowej z wykorzystaniem niniejszego szablonu należy pamiętać o obowiązku zachowania zgodności prezentowanych treści z zawartością odpowiedniej karty tematu (merytorycznie musi się wszystko zgadzać).

% Zasadniczo tematy prac inżynierskich wiążą się z wykonaniem jakiegoś konkretnego dzieła (produktu). Formułując cel używa się zwrotów:
% budowa, implementacja, projekt, przeprowadzenia itp. Rolą dyplomanta (na kierunku informatyka) jest dostarczenie dzieła (przynajmniej w~formie prototypu, spełniającego podstawowe wymagania funkcjonalne). Niniejszy szablon pozwolić ma na opisanie tego dzieła.

% Nieco inaczej jest w przypadku tematów prac magisterskich. Tutaj temat wraz z opisem celu i zakresu wyznaczać mają jakiś kierunek badań czy analiz. Od razu nie wiadomo przecież, do czego się dojdzie. A jeśli byłoby wiadomo, to nie byłoby sensu robić badań.  Na tym właśnie polega "piękno" pracy badawczej. 
% Zwykle więc podczas formułowania tematów i opisów tego typu prac stosuje się słowa: badanie, analiza, przegląd, charakterystyka itp. Rolą magistranta jest eksploracja wyznaczonego kierunku, dobre uwarunkowanie analizowanych problemów, przeprowadzenie badań, dostarczenie odpowiedzi. Niniejszy szablon w tym przypadku posłużyć ma za ramy, dzięki którym praca może przyjąć formę dokumentu naukowego.

% Po kompilacji wynikowy dokument \texttt{Dyplom.pdf} należy załadować do systemu APD USOS (\url{https://apd.usos.pwr.edu.pl/}) celem weryfikacji antyplagiatowej i dalszego procesowania. System ten zmienia nazwę załadowanemu plikowi na taką, w której ukryty jest kod wydziału, kod kierunku, typ pracy i jeszcze parę innych danych. Nazwa ta przyjmuje, przykładowo, następującą postać: \texttt{W4N\_\#\#\#\#\#\#\_W04-ITE-INZ\_W04-ITEP-000P-OSIW7.pdf}, gdzie \texttt{\#\#\#\#\#\#} to miejsce na numer indeksu dyplomanta. Z tej racji trudno przewidzieć, jak ostatecznie należy się odwoływać do pliku zarejestrowanego w systemie. Jest to o tyle istotne, iż podczas składania prac dyplomowych wciąż stosowaną praktyką jest dołączanie płyty CD/DVD z jej wynikami (zawierającej dokumentem \texttt{pdf} z tekstem pracy, jak również kodami źródłowymi stworzonego dzieła, instalatorami itp.). Proszę zajrzeć do dodatku \ref{chap:opis-plyty} po dodatkowe wyjaśnienia.



% W rozdziale pierwszym przedstawiono w zarysie czym jest i czego dotyczy niniejszy dokument (jest to szablon, który można zastosować podczas redagowania pracy dyplomowej inżynierskiej bądź magisterskiej). W rozdziale drugim opisano sposób pracy z szablonem. W kolejnym, trzecim rozdziale, przedstawiono zalecenia dotyczące formatowania dokumentu. Rozdział ten pełni rolę czysto informacyjną (dostarczony szablon zapewnia uzyskanie opisanego tam formatowania).
% W rozdziale czwartym zwrócono uwagę na redakcję pracy dyplomowej (od strony edytorskiej i merytorycznej).
% Rozdział piąty poświęcono na uwagi techniczne. Ostatni, szósty rozdział, przeznaczono na kilka słów podsumowania oraz ,,lorem ipsum'' -- wygenerowany tekst, pełniący rolę ,,wypełniacza'', wykorzystany w celach poglądowych (jak dzielić dokument na sekcje).
% Pracy towarzyszy przykładowy wykaz literatury oraz przykładowe dwa dodatki. 








